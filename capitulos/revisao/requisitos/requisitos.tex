\section{Requisitos de segurança para eleições}

Na história, problemas abrangendo desde erro humano, inconsistência nos resultados e tentativas de fraude atingiram diversas eleições, mesmo depois da digitalização do voto e dos constantes aprimoramentos dos sistemas utilizados. É claro que uma eleição não depende apenas da segurança do processo, pois existem requisitos de usabilidade e desempenho do sistema que serão considerados em qualquer sistema eleitoral que aspire ser amplamente adotado.

Levando em consideração os possíveis objetivos de uma eleição, destacam-se caraterísticas desejadas no processo, em especial quanto à integridade do resultado, o sigilo do voto, a autenticação dos votantes, a abrangência da eleição e a disponibilidade do sistema~\cite{jones2012broken}.

Ao se pensar em segurança para as eleições de mais alta ordem, como eleições presidenciais, é importante salientar que o sistema deve ter o mínimo possível de requisitos de entidades confiáveis, uma vez que isso criaria mais pontos de falha de requisitos. Neste ponto, podemos salientar que o governo, os agentes e tribunais eleitorais, os políticos e até os próprios eleitores são corruptíveis, além de todos os componentes da rede que podem ser manipulados, e isso tem de ser contado para que o sistema atenda aos requisitos de segurança.

\subsection{Integridade do resultado}

Uma definição curta da integridade do resultado de uma eleição democrática é de que o vencedor é o candidato escolhido pela intenção de voto do povo~\cite{norris2014electoral}. A integridade do resultado está diretamente relacionada com a verificabilidade do processo.

Em termos simples, o vencedor é o candidato com a maior quantidade de votos, mas em eleições práticas as dificuldades tornam quase impossível a contabilização perfeita de votos. A precisão do resultado depende do sistema, e é uma das áreas em que a modernização por meio do uso de computadores pode aumentar a confiabilidade de uma eleição.

No que diz respeito à intenção do voto, que significa que o voto deve ser contabilizado para o candidato que o eleitor tem intenção de eleger. Isso não acontece em uma eleição cujos resultados são manipulados, como acontece em alguns países emergentes com governos autoritários~\cite{van2013elections}. Também pode acontecer por uma falha em algum processo, como a recontagem de votos da eleição presidencial da Flórida em 2000, onde o desenho da cédula causou confusão.
\subsection{Sigilo do voto}
\section{Cadastro e autenticação de eleitores}

O subsistema de cadastro e autenticação de eleitores é responsável por
registrar os dados e confirmar a identidade dos eleitores que acessem o sistema
de votação. Nos sistemas citados no \autoref{ch:trabalhos correlatos}, são
utilizados diferentes sistemas de registro e autenticação, notadamente o uso de
\textit{tokens} enviados para os eleitores antes da votação.

O Brasil já possui alguns sistemas que utilizam de autenticação eletrônica, na
forma de certificados digitais, como o e-CNPJ. Esse tipo de autenticação provê
muito mais segurança do que assinaturas físicas, pois anula a possibilidade de
falsificação de identidade, ou seja, garantidamente o portador de um
certificado digital é o seu dono e não um impostor.

Essa modalidade de autenticação é bastante flexível, sendo utilizada também por
bancos para autenticar os dispositivos utilizados pelos clientes, sejam
celulares, tablets e similares, sem a necessidade de senhas ou \textit{tokens},
mas é possível combinar com estes para prover ainda mais segurança. Idealmente,
deve-se utilizar mais de um fator de autenticação para evitar clonagem ou roubo
das credenciais, sendo bastante comum na internet os \textit{tokens} baseados
no tempo (\textit{time-based one-time password} ou \textbf{TOTP}).

O sistema Helios tem suporte a \textit{plugins} de autenticação, suportando
padrões abertos de \textit{single sign-on} como CAS, OAuth, Shibboleth e
sistemas de identificação federados como o OpenID Connect através destes. Pode
também, teoricamente, suportar autenticação através de \textit{smart cards} e
outras formas de certificados digitais, mas seu repositório não integra nenhuma
destas soluções de autenticação citadas atualmente.

A terceirização do serviço de autenticação permite delegar a confiabilidade
deste para outra autoridade dedicada. Por exemplo, diversas universidades
possuem serviços de autenticação centralizada utilizando algum dos protocolos
citados anteriormente, e a própria entidade garante a autenticidade dos seus
registros e a autorização do seu uso. Isso reduz a superfície de ataque do
sistema de eleição e o torna mais flexível e adaptável a novas tecnologias.

Para a implementação de referência desenvolvida neste trabalho, optou-se por
utilizar o protocolo OpenID Connect~\cite{sakimura2014openid}, utilizando como
servidor de autorização o Google. Desta forma, a autenticação de usuários é
realizada por um terceiro, e o servidor do registro de votos não gerencia os
usuários, se limitando a armazenar apenas o nome e e-mail.

Utiliza-se também a tecnologia \textit{Credential Management API}
(\textbf{CM-API}) para armazenar e recuperar credenciais federadas diretamente
no \textit{browser}, simplificando o fluxo de reautenticação de uma sessão
expirada~\cite{Credenti82:online}. Essa tecnologia ainda é considerada
experimental, estando disponível somente no navegador Chrome no momento de
escrita deste trabalho.

A combinação destas duas tecnologias permite armazenar o mínimo possível de
informação sobre o eleitor no servidor de eleição, delegando a tarefa de
autenticação para outros serviços de confiança do próprio usuário, como o
navegador de sua preferência e o serviço de autenticação a ser consultado.

A \textbf{CM-API} também prevê suporte para autenticação por senha e por chave
pública. Para a autenticação por chave pública, é possível utilizar
dispositivos de chave assimétrica, como \textit{smart cards} contendo
certificados digitais para autenticação física dos usuários, se este nível de
segurança for desejado. Nenhum navegador suporta este método no momento da
escrita deste trabalho, mas Chrome e Firefox já possuem implementações
parciais.

\subsection{Abrangência da eleição}
\subsection{Disponibilidade do sistema}