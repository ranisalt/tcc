\subsection{Integridade do resultado}

Uma definição curta da integridade do resultado de uma eleição democrática é de que o vencedor é o candidato escolhido pela intenção de voto do povo~\cite{norris2014electoral}. A integridade do resultado está diretamente relacionada com a verificabilidade do processo.

Em termos simples, o vencedor é o candidato com a maior quantidade de votos, mas em eleições práticas as dificuldades tornam quase impossível a contabilização perfeita de votos. A precisão do resultado depende do sistema, e é uma das áreas em que a modernização por meio do uso de computadores pode aumentar a confiabilidade de uma eleição.

No que diz respeito à intenção do voto, que significa que o voto deve ser contabilizado para o candidato que o eleitor tem intenção de eleger. Isso não acontece em uma eleição cujos resultados são manipulados, como acontece em alguns países emergentes com governos autoritários~\cite{van2013elections}. Também pode acontecer por uma falha em algum processo, como a recontagem de votos da eleição presidencial da Flórida em 2000, onde o desenho da cédula causou confusão.