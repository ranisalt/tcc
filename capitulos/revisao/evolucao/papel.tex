\subsection{Urna e cédula de papel}

Posterior ao sistema de aclamação, quando governos e eleições se tornaram mais
robustos, adotou-se a técnica da cédula de papel para registrar os votos. Nele,
o eleitor insere uma cédula em uma urna lacrada e, ao final da eleição, a urna
é aberta e os votos são contabilizados. A praticidade e as evoluções desde
então fazem com que seja utilizado até hoje em diversas eleições.

Esse sistema provê uma forma simples de sigilo do voto e aprimora a integridade
do processo, mas a adição de novos elementos (a urna e a cédula) abriu novas
possibilidades de ataques e fraudes. As fraudes elaboradas~\nocite{leslie1856}
motivaram o desenvolvimento de urnas mais seguras, como as urnas de vidro, que
permitiam que os eleitores observassem o estado da urna.

\subsubsection{Cédulas impressas}

Além da segurança das próprias urnas, o desenho das cédulas também evoluiu para
comportar novos níveis de segurança e acessibilidade das votações. Novas
cédulas que continham nomes dos candidatos impressos para escolha do eleitor
foram instauradas na Austrália, no século XIX.

Embora facilite a escolha por parte do eleitor ao apresentar de antemão os
candidatos, em contraste com o preenchimento das cédulas vazias, o desenho da
cédula afeta diretamente a intenção de voto e a intuitividade do
processo~\cite{everett2006measuring}.

A versão impressa exige um nível mínimo de alfabetização do eleitor, uma vez
que este precisa identificar qual candidato deve marcar, que pode restringir a
abrangência da eleição, ainda que a cédula utilize símbolos em vez de nomes.
Afeta também o custo do processo, pois o governo precisa arcar com os custos da
confecção, impressão e distribuição das cédulas.

Para aumentar a privacidade e sigilo, os locais de votação frequentemente
utilizam uma cabine onde o eleitor pode, fora da visão pública, preencher sua
cédula antes de introduzí-la na urna.

As eleições brasileiras demoraram a adotar as técnicas de sigilo do voto para
cédulas. A classe política foi contrária ao voto secreto, argumentando que o
cidadão deveria ser responsável por seu voto. Legalmente, o voto secreto chegou
a ser proibido~\cite{nicolau2012eleicoes}.

\subsubsection{Voto encadeado}

Um método de fraude bastante elaborado do sistema de cédulas e urnas é o voto
encadeado, onde um adversãrio pode exercer coerção e comprar votos mesmo com a
privacidade das cabines de votação.

Esse esquema requer que um adversário que previamente tenha acesso a uma cédula
em branco, preencha com o candidato para o qual queira fraudar votos, entregue
para um eleitor legítimo e exige deste sua cédula vazia em troca.

Assumindo que o mesário entregue apenas uma cédula para o eleitor e exija que
uma cédula seja inserida na urna, isso significa que o eleitor depositou a
cédula do adversário. O processo é repetido, fraudando votos de forma
indetectável por utilizar cédulas legítimas.

Para mitigar este ataque, é possível numerar as cédulas e, antes de inserí-las
na urna, conferir se o número a ser inserido é da cédula que foi entregue ao
eleitor. A numeração deve ser destacável e removida antes de inserir na urna,
para impossibilitar a distinção de dois votos e o rompimento do sigilo.
