\subsection{Sensores óticos}

O uso de sensores óticos para registro de votos é semelhante aos sistemas de
testes padronizados, como o ENEM, e foram usados pela primeira vez em
experimentos eleitorais na década de 1960. Apesar da tabulação ser semelhante
aos cartões perfurados, é mais intuitivo justamente pelo amplo uso em outras
situações.

Diferentes tecnologias foram usadas como sensores óticos em eleições.
Inicialmente, sensores simples contabilizavam marcas no papel da cédula que
fossem mais escuras que um dado limite, enquanto modelos mais modernos utilizam
algoritmos de reconhecimento de imagens com resoluções altas.

Os modelos de máquinas de tabulação poderiam tanto fazer a contagem no local da
votação quanto em centrais de contagem de votos, para onde cédulas são levadas
após a votação.

Como os outros equipamentos que dependem de \textit{software}, o uso de
sensores óticos é suscetível a ataques que ganhem acesso ao sistema operacional
da máquina. A cédula é um documento e pode ser utilizada como comprovação de
voto numa fraqueza de coerção. Além disso, assim como em cédulas de papel
simples, pode sofrer de voto encadeado.
