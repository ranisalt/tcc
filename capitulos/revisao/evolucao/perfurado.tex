\subsection{Cartão perfurado}

A tecnologia de cartões perfurados foi adaptada para eleições na década de
1960, pela empresa IBM, e aprimorada por professores da Universidade da
Califórnia em Berkeley.

Neste método, o cartão perfurado entregue ao eleitor possui espaços para
destaque de acordo com cada candidato possível. O eleitor insere o cartão em
uma máquina com um catálogo dos candidatos, e deve perfurar o destaque
correspondente com o seu voto. A máquina pode ser portátil ou embutida na
cabine de votação.

Os problemas com os cartões perfurados apareceram logo após sua introdução em
eleições. Pedaços destacados dos cartões se acumulavam nas máquinas e causavam
votos indevidos. Este problema causou notável discussão nas eleições dos
Estados Unidos em 2000~\cite{leib2002florida}, sobre se uma área parcialmente
destacada deveria ser computada.

Ao contrário das cédulas de papel mais modernas, é mais difícil recontar
manualmente os cartões perfurados. Enquanto é fácil verificar uma marca no
papel, um destaque incompleto pode ser resultado de hesitação do eleitor que
não deseja aquele voto, ou da dificuldade de perfurar o cartão por parte do
eleitor que deseja o voto~\cite{roth1998disenfranchised}.

Uma derivação dos cartões perfurados, o DataVote traz impresso os nomes dos
candidatos no cartão, de forma que o eleitor destaca seu voto diretamente e não
é necessária a máquina para destaque dos votos. Desta forma, resolve o problema
dos pedaços que atrapalham a realização correta dos votos, e diminui a confusão
do eleitor pois não exige o alinhamento exato do cartão na
máquina~\cite{cranor2003search}.
