\subsection{Urnas eletrônicas}

Seguindo as tendências da tecnologia, os sistemas eleitorais também foram
computadorizados no fim do Século XX. No início do seu uso, a euforia ofuscou a
discussão necessária sobre a segurança, privacidade e confiabilidade das urnas
eletrônicas, mesmo depois de sua implementação nacional no Brasil e dos
problemas encontrados em outros países.

Em análise realizada por \textcite{kohno2004analysis}, destaca-se que as
empresas que desenvolveram os primeiros terminais eletrônicos de votação o
fizeram de forma que robustez, segurança e confiabilidade das eleições depende
inteiramente do \textit{software} sendo executado pelos terminais. Se estes
terminais pudessem ser manipulados, seja por eleitores ou desenvolvedores
inescrupulosos, ou até pelo sistema operacional utilizado, a precisão dos
resultados é posta em xeque.

Nos Estados Unidos, a lei federal denominada \textit{Help America Vote Act}
(HAVA), de 2002, tinha por objetivo incentivar a substituição dos sistemas de
cartão perfurado e máquinas de alavancas por equipamentos mais modernos e
estabelecer requisitos de acessibilidade aos locais de votação, o que tornava
as urnas eletrônicas o único tipo de equipamento que os
satisfazia~\cite{davis1996direct}.

As urnas eletrônicas são amplamente criticadas por uso incorreto de funções
criptográficas e baixa qualidade de código, respaldadas por operarem código
proprietário e fechado em geral, e possibilidade de fraude em cartões de
configuração, concentrando desconfiança de estudiosos da
área~\cite{feldman2006security}.

A solução então proposta é a do voto impresso\index{voto impresso}, formalmente
conhecido como \textit{voter-verifiable audit trail} (rastro de auditoria
verificável pelo eleitor). A parte mais importante é a de verificação pelo
eleitor: se a máquina não gera uma impressão correta do voto, o papel é
destruído mecanicamente e não é inserido na urna de auditoria. Desta forma, o
funcionamento do terminal de votação não é mais o único ponto de falha, pois
pode ser removido de circulação se estiver produzindo resultados incorretos.

Esse tipo de vulnerabilidade é mais perigosa que as vulnerabilidades dos
sistemas tradicionais de cédulas porque são sutis, imperceptíveis da visão do
eleitor e das autoridades, e podem acontecer por mero descuido dos
responsáveis. Como descoberto por \textcite{feldman2006security}, um modelo de
urna eletrônica utilizada nos Estados Unidos poderia ser manipulada inserindo
um cartão de memória com código malicioso em menos de um minuto, situação
plausível uma vez que os operadores das urnas frequentemente as operam sem
supervisão.

No discurso popular de tempos recentes, existem críticas ferrenhas ao modelo
atual das urnas eletrônicas, das quais já foram encontradas inúmeras brechas de
segurança nos últimos testes públicos, mesmo estes testes sendo executados sob
forte vigilância do tribunal responsável. Em paralelo, escândalos de corrupção,
compra de votos e afins apenas dão razão aos desafetos da evolução tecnológica
do nosso sistema eleitoral.

Dentre os poucos relatórios de agentes externos sobre os componentes internos
da urna eletrônica, certamente o mais conhecido e completo é o de
\textcite{aranha2012vulnerabilidades}, que apontou diversos problemas de
segurança no \textit{software} executado na urna e também na maneira como o
órgão responsável modelou os potenciais vetores de ataque ao sistema.

A situação não é melhor em implementações utilizadas por outros países além do
Brasil. Na maior democracia do mundo, a Índia, uma louvável iniciativa do
governo instalou milhões de urnas eletrônicas de baixo custo em todo o país,
até mesmo nas regiões mais remotas, para maximizar o sufrágio universal.
Durante muito tempo, mesmo após a literatura amplamente depreciar as urnas sem
voto impresso, o governo insistia que as urnas eram completamente seguras.

Na rigorosa revisão por \textcite{wolchok2010security}, relata-se que tanto o
\textit{software} quanto o \textit{hardware} das urnas utlizadas na Índia pode
ser manipulado por um agressor com acesso ao dispositivo. Como as peças são
construídas por terceiros, dificilmente um componente com um comportamento
malicioso seria detectado na montagem da urna eletrônica, muito menos por um
operador leigo. Além disso, o \textit{software} executado também é gravado no
\textit{chip} por um terceiro, que poderia instalar uma versão adulterada que
dificilmente levantaria suspeitas.

Por fim, pela simplicidade da construção, um agressor poderia tentar substituir
uma urna por um modelo adulterado, ou substituir componentes internos da urna
por versões maliciosas, considerando que não há uma maneira prática para que
operadores verifiquem a autenticidade das urnas sendo utilizadas.

Praticamente todas as democracias do mundo que utilizam urnas eletrônicas
\index{países que usam urnas eletrônicas} e formas similares de
terminais de votação, notadamente Venezuela, Países Baixos, Alemanha, Bélgica,
Rússia e Equador, além dos já citados Estados Unidos e Índia, já substituíram
urnas eletrônicas sem confirmação impressa do voto, agora chamadas de máquinas
de primeira geração. O único país que ainda permite máquinas de primeira
geração é o Brasil, que está em processo de substituição por máquinas de
segunda geração, com confirmação impressa.
