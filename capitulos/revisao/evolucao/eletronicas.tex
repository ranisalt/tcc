\subsection{Urnas eletrônicas}

Seguindo as tendências da tecnologia, os sistemas eleitorais também foram computadorizados no fim do Século XX. No início do seu uso, a euforia ofuscou a discussão necessária sobre a segurança, privacidade e confiabilidade das urnas eletrônicas, mesmo depois de sua implementação nacional no Brasil e dos problemas encontrados em outros países.

Nos Estados Unidos, a lei federal denominada \textit{Help America Vote Act} (HAVA), de 2002, tinha por objetivo incentivar a substituição dos sistemas de cartão perfurado e máquinas de alavancas por equipamentos mais modernos e estabelecer requisitos de acessibilidade aos locais de votação, o que tornava as urnas eletrônicas o único tipo de equipamento que os satisfazia~\cite{davis1996direct}.

As urnas eletrônicas são amplamente criticadas por uso incorreto de funções criptográficas e baixa qualidade de código, respaldadas por operarem código proprietário e fechado em geral, e possibilidade de fraude em cartões de configuração, concentrando desconfiança de estudiosos da área~\cite{kohno2004analysis,feldman2006security}.