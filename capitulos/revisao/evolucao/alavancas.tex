\subsection{Máquinas de alavancas}

Máquinas de votação por alavancas foram usadas inicialmente nos Estados Unidos
e desenhadas para fazer o processo de votação ser simples e sigiloso. No
início, eram o aparato mais tecnológico existente, com muitas partes móveis. Ao
se utilizar uma máquina para votar mecanicamente, se evita o problema de
interpretação da cédula. Além disso, alguns
modelos~\nocite{gillespie1899voting} obrigavam que o eleitor estivesse atrás de
cortinas, mantendo a privacidade do seu voto.

Como todo aparato mecânico, as máquinas de alavancas eram suscetíveis a falhas
e travamentos das peças, e eram especialmente afetadas justamente pela grande
quantidade de partes móveis que as faziam máquinas complexas e avançadas. Uma
falha no mecanismo de contagem poderia afetar a contabilização de todos os
votos de um candidato.

Nas máquinas com contadores não havia registro individual dos votos, portanto,
estas poderiam ser manipuladas por eleitores e técnicos de manutenção para
propositalmente falhar na contagem de votos, por vezes de formas
indetectáveis~\cite[p.~42-45]{jones2012broken}. Já nos modelos com impressão em
rolo de papel era possível mapear cada voto para cada eleitor, pois eram
impressos em ordem~\cite[p.~26]{jones2012broken}.

As máquinas causaram uma mudança de paradigma das fraudes. A forma mais comum
de fraudes com cédulas, de estufar as urnas ou encadear os votos, não era mais
possível. No entanto, era igualmente fácil corromper as próprias autoridades
eleitorais e, como o eleitor não consegue confirmar que seu voto foi
corretamente contado, as fraudes se tornavam mais silenciosas.
