\subsection{Voto por aclamação ou \textit{viva voce}}

O primeiro sistema de votação, e tecnologicamente o mais simples, foi
implementado nos primórdios da democracia na Grécia Antiga, há quase 3 mil
anos. Nele, o eleitor se dirigia aos delegados da eleição e se manifestava em
que candidato tinha intenção de eleger~\cite{wolfson1899ballot}. Esse tipo de
votação tinha múltiplos pontos de falha, muito por sua simplicidade, e aos
valores morais e éticos dos responsáveis.

O sistema não possui nenhum mecanismo de privacidade, visto que o voto é
público, e não oferece nenhuma resistência à coerção ou venda de
voto~\cite{buchstein2010public}. Por outro lado, esse sistema de votação
permite uma perfeita auditoria do resultado, mesmo não havendo garantias de que
as autoridades ajam de forma honesta com o resultado da eleição.

A autenticação dos votantes também era problemática na época do uso desse
sistema. Como a tecnologia era muito primordial, não haviam grandes registros
centralizados de população e formas de comprovar cidadania, de forma que até
meados do século XIX os eleitores juravam sobre a bíblia que poderiam votar e o
fariam apenas uma vez, sujeitos apenas à sua crença religiosa.

No Brasil, um processo baseado no \textit{viva voce}, onde o votante falava
``em segredo'' ao juiz responsável foi utilizado durante o período
imperial~\cite{nicolau2012eleicoes}. Este curioso sistema de certa forma
reduzia o problema da falta de privacidade, mas impedia qualquer forma de
auditoria pública do resultado.

O sistema de voto por aclamação não é prático em larga escala, por isso,
governos que adotavam este modelo o aplicavam em diferentes níveis (ou graus)
de votação indireta. A população de um distrito escolhia o seu representante no
colégio eleitoral, que então votaria nos parlamentares em outra instância da
eleição. Sistemas indiretos por aclamação foram utilizados na França, Espanha e
Portugal, além do Brasil~\cite{nicolau2012eleicoes}.
