\subsection{Scantegrity}

O sistema Scantegrity, descrito em \textcite{chaum2008scantegrity}, consiste em
uma camada de segurança para sistemas de sensores óticos que implementa
auditabilidade fim-a-fim dos resultados. Isso é obtido por meio de códigos de
confirmação que permitem o eleitor conferir que seu voto está incluso no
resultado, sem oferecer outras informações sobre o voto.

É uma evolução do sistema Punchscan, desenvolvido pela mesma equipe junto com
outros contribuidores.

Uma versão aprimorada, chamada Scantegrity II, desenvolvida pelos mesmos
autores, utiliza tinta invisível para obter melhorias na usabilidade e na
resolução de disputas~\cite{chaum2008scantegrityII}. É possível verificar o
resultado de uma eleição de forma independente de software pelo seu uso de
técnicas criptográficas.

O eleitor tem a confirmação da contagem do seu voto por meio de códigos
revelados com uma tinta invisível no preenchimento da cédula. A aleatoriedade
destes códigos também provê resistência à coerção e garante o segredo do voto
mesmo que sejam revelados.

Um teste público foi realizado em 2009 na cidade estadunidense de Takoma Park,
em Maryland, sob a hipótese de que o Scantegrity seria um sistema confiável,
fácil de usar e administrar, seria aceito e confiado como alternativa de alta
confiança e que os eleitores confeririam seus votos
posteriormente~\cite{sherman2010scantegrity}.

O teste revelou que o Scantegrity tem um tempo médio por eleitor de 167
segundos, o dobro do tempo médio das eleições gerais do Brasil em urnas
eletrônicas, que varia entre 61 e 85 segundos~\cite{tse2014tempomedio}. Apesar
de considerarem que o sistema foi fácil de usar, muitos eleitores não usaram
corretamente as ferramentas de privacidade fornecidas e não entenderam que
poderiam auditar seu voto online.

Além dos eleitores, os mesários também responderam ter dificuldades com o uso
do sistema, alegando que havia muita informação e questionando se as melhorias
do Scantegrity valiam a pena pela dificuldade adicional.
