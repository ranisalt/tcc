\subsection{Punchscan}

O sistema Punchscan, descrito em~\cite{fisher2006punchscan}, é um sistema
híbrido entre cédulas e dispositivos eletrônicos, que utiliza sensores óticos,
implementa auditabilidade fim-a-fim dos resultados e emite um recibo de voto ao
eleitor.

Apesar de utilizar um software para contagem de votos, o Punchscan é
independente deste. Sua segurança é baseada em premissas criptográficas em vez
de depender da segurança do software, como nas urnas eletrônicas, e por isso
pode ser executado em sistemas operacionais não confiáveis e ainda manter
integridade incondicional.

A cédula do Punchscan é dividida em duas camadas. Na camada superior, os
candidatos são listados com um símbolo ao lado do nome, e abaixo há uma série
de furos redondos. Por estes furos, na camada inferior, estão os símbolos
correspondentes aos candidatos.

Para efetuar um voto, o eleitor deve encontrar o furo com o símbolo
correspondente ao candidato desejado e marcar com uma caneta especial que é
propositalmente maior que o furo. O eleitor então separa as camadas, escolhe
uma das duas para guardar como recibo e destrói a outra. O recibo é escaneado
por um sensor ótico para tabulação do resultado.

A ordem dos candidatos na camada superior é disposta de forma pseudoaleatória,
assim como a dos símbolos na camada inferior. Dessa forma, o recibo não contém
informação suficiente para determinar o voto: se a camada superior é escolhida,
a ordem dos símbolos nos furos é desconhecida; se a camada inferior é
escolhida, a ordem dos símbolos referentes aos candidados é desconhecida.

Portanto, o Punchscan é resistente a coerção, uma vez que o eleitor não
consegue provar em quem ele votou.
