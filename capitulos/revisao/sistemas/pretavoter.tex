\subsection{Prêt à Voter}

O sistema Prêt à Voter, descrito em \textcite{ryan2009pret}, é um sistema que procura garantir a precisão da contagem e a privacidade do voto de forma independente do software e hardware utilizado no processo. Em particular, este sistema permite que eleitores confirmem seu voto no resultado final, e previne a coerção e compra de votos.

No Prêt à Voter, a lista de candidatos na cédula é ordenada de acordo com uma variável criptográfica, como no Punchscan. Isso garante o segredo do voto no recibo e mitiga qualquer tendência a favor do candidato no topo da lista como pode ocorrer em ordens fixas.

É um sistema com auditabilidade fim-a-fim, e os eleitores podem conferir que seu voto foi contabilizado corretamente nos resultados públicos da votação.

Diversas modificações foram feitas ao sistema para adicionar novas características, como métodos de verificação simplificados para eleitores leigos~\cite{lundin2008human}, aprimoramentos no sistema criptográfico~\cite{ryan2006pret} e um híbrido entre Punchscan e Prêt à Voter elaborado para eleições por correios~\cite{popoveniuc2007simple}.