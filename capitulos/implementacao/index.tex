\chapter{Implementação}
\label{ch:implementação}

Na elaboração de um novo protocolo e sistema de eleição, alguns desafios devem
ser superados ou considerados para garantir a integridade dos resultados.
Adotando a mentalidade de segurança, rapidamente encontraríamos os problemas
dos sistemas mais primitivos citados anteriormente, como a falta de privacidade
da votação por aclamação e as possibilidades de adulteração em urnas
eletrônicas.

Neste trabalho, foi desenvolvida uma especificação de um sistema remoto de
eleição, composto de um protocolo de procedimento de eleição baseado no ADDER,
da descrição da interação com serviços terceirizados de autenticação de
usuários e de uma biblioteca de primitivas criptográficas, que foi desenvolvida
para uso neste sistema.

\section{Cadastro e autenticação de eleitores}

O subsistema de cadastro e autenticação de eleitores é responsável por
registrar os dados e confirmar a identidade dos eleitores que acessem o sistema
de votação. Nos sistemas citados no \autoref{ch:trabalhos correlatos}, são
utilizados diferentes sistemas de registro e autenticação, notadamente o uso de
\textit{tokens} enviados para os eleitores antes da votação.

O Brasil já possui alguns sistemas que utilizam de autenticação eletrônica, na
forma de certificados digitais, como o e-CNPJ. Esse tipo de autenticação provê
muito mais segurança do que assinaturas físicas, pois anula a possibilidade de
falsificação de identidade, ou seja, garantidamente o portador de um
certificado digital é o seu dono e não um impostor.

Essa modalidade de autenticação é bastante flexível, sendo utilizada também por
bancos para autenticar os dispositivos utilizados pelos clientes, sejam
celulares, tablets e similares, sem a necessidade de senhas ou \textit{tokens},
mas é possível combinar com estes para prover ainda mais segurança. Idealmente,
deve-se utilizar mais de um fator de autenticação para evitar clonagem ou roubo
das credenciais, sendo bastante comum na internet os \textit{tokens} baseados
no tempo (\textit{time-based one-time password} ou \textbf{TOTP}).

O sistema Helios tem suporte a \textit{plugins} de autenticação, suportando
padrões abertos de \textit{single sign-on} como CAS, OAuth, Shibboleth e
sistemas de identificação federados como o OpenID Connect através destes. Pode
também, teoricamente, suportar autenticação através de \textit{smart cards} e
outras formas de certificados digitais, mas seu repositório não integra nenhuma
destas soluções de autenticação citadas atualmente.

A terceirização do serviço de autenticação permite delegar a confiabilidade
deste para outra autoridade dedicada. Por exemplo, diversas universidades
possuem serviços de autenticação centralizada utilizando algum dos protocolos
citados anteriormente, e a própria entidade garante a autenticidade dos seus
registros e a autorização do seu uso. Isso reduz a superfície de ataque do
sistema de eleição e o torna mais flexível e adaptável a novas tecnologias.

Para a implementação de referência desenvolvida neste trabalho, optou-se por
utilizar o protocolo OpenID Connect~\cite{sakimura2014openid}, utilizando como
servidor de autorização o Google. Desta forma, a autenticação de usuários é
realizada por um terceiro, e o servidor do registro de votos não gerencia os
usuários, se limitando a armazenar apenas o nome e e-mail.

Utiliza-se também a tecnologia \textit{Credential Management API}
(\textbf{CM-API}) para armazenar e recuperar credenciais federadas diretamente
no \textit{browser}, simplificando o fluxo de reautenticação de uma sessão
expirada~\cite{Credenti82:online}. Essa tecnologia ainda é considerada
experimental, estando disponível somente no navegador Chrome no momento de
escrita deste trabalho.

A combinação destas duas tecnologias permite armazenar o mínimo possível de
informação sobre o eleitor no servidor de eleição, delegando a tarefa de
autenticação para outros serviços de confiança do próprio usuário, como o
navegador de sua preferência e o serviço de autenticação a ser consultado.

A \textbf{CM-API} também prevê suporte para autenticação por senha e por chave
pública. Para a autenticação por chave pública, é possível utilizar
dispositivos de chave assimétrica, como \textit{smart cards} contendo
certificados digitais para autenticação física dos usuários, se este nível de
segurança for desejado. Nenhum navegador suporta este método no momento da
escrita deste trabalho, mas Chrome e Firefox já possuem implementações
parciais.

\section{Registro público de votos}
\label{sect:registro}

O registro de votos é o subsistema responsável por receber e armazenar com
segurança os votos dos eleitores. Este sistema não deve ter capacidade de abrir
os votos depositados, não tendo acesso à chave privada utilizada para
decifrá-los. Dessa forma, sua responsabilidade é evitar votos duplicados,
permitindo que um eleitor sobrescreva seu voto, e reportar para o sistema de
contagem o resultado final.

O subsistema de registro não deve ter acesso à chave privada de nenhuma
eleição, para garantir o sigilo do voto. Deve somente fornecer uma interface
para a inserção e conferência dos votos, como faz uma urna, repassando o
somatório cifrado às autoridades de contagem ao final do processo.

Pode ser implementado como uma camada fina de software sobre um banco de dados
que confere a autenticação do usuário com o subsistema de autenticação,
armazena os votos válidos de forma cifrada neste banco de dados e provê uma
forma de visualização dos votos já inseridos, sem informações sobre os
eleitores.


\subimport{primitivas/}{index}

\section{Protocolo de uma eleição}

O protocolo de uma eleição utilizado no ADDER, e portanto no qual é inspirado
este trabalho, se concentra em torno do registro digital público de votos,
chamado de \textit{bulletin board} ou \textit{ledger}, e está descrito em
\textcite{benaloh1987verifiable}. Este registro consiste em um mural público
onde todos os usuários antenticados, tanto eleitores, autoridades e
administradores podem inserir dados.

Na implementação, o \textit{ledger} consiste em um banco de dados relacional
acessado através de uma interface \textit{web} que armazena as credenciais de
uma eleição e versões criptografadas dos votos, e um servidor de autenticação
que emite \textit{tokens} com os quais os usuários podem se autenticar com o
\textit{ledger} e realizar ações de acordo com sua permissão. Uma interface
\textit{web} acessa os dados destes servidores de forma transparente.

O procedimento de execução de uma eleição é dividido em passos sequenciais para
criar as informações necessárias, receber os votos e calcular o resultado
final.

Para criar uma nova eleição, um administrador do sistema envia os parâmetros da
nova eleição para o \textit{ledger} através da interface \textit{web}. Estes
parâmetros são os identificadores da eleição, a lista de $n$ autoridades, a
lista de eleitores, a quantidade mínima $m_{min}$ e máxima $m_{max}$ de
candidatos por voto, a lista de candidatos, o limite mínimo de autoridades $t$
e os horários limites para os passos seguintes. Estas informações ficam
publicamente disponíveis no \textit{ledger} para conferência.

No segundo passo, as autoridades devem submeter suas chaves públicas para o
sistema, para que toda a comunicação de informação sensível entre o sistema e a
autoridade se dê por forma criptografada, nunca em texto claro. A autoritade
também deve submeter um desafio para provar que detém a chave privada
relacionada à chave pública enviada. Se menos de $t$ autoridades enviarem
chaves públicas até o prazo expirar para este passo, a eleição é encerrada com
falha.

No terceiro passo, o administrador que criou a eleição solicita que o sistema
gere o par de chaves da eleição. Para isso, utiliza o algoritmo de geração de
chaves descrito em \textcite{fouque2000sharing}, que cria uma chave pública,
divide a chave privada entre as autoridades que participaram do passo anterior
e envia cada parte por e-mail de forma criptografada. A chave pública e as
chaves de verificação são publicados no \textit{ledger} junto com os outros
detalhes da eleição gerados até agora. Se menos de $t$ autoridades confirmarem
o recebimento de suas chaves privadas parciais até o prazo expirar para este
passo, a eleição é encerrada com falha.

Antes do quarto passo, que é a votação de fato, o administrador deve publicar a
lista de eleitores no \textit{ledger}. Os eleitores são identificados
unicamente por e-mail, e como especificado anteriormente, são autenticados por
um servidor de autenticação externo.

Durante a votação, somente os eleitores submetidos anteriormente podem inserir
seus votos no \textit{ledger}. Os votos são cifrados e acompanhados de uma
prova de formação correta, de forma que o eleitor pode conferir se seu voto
cifrado está corretamente publicado e o sistema pode garantir que o eleitor
votou em pelo menos $m_{min}$ opções e em até $m_{max}$ opções. O eleitor pode
substituir seu voto quantas vezes desejar, de forma a mitigar tentativas de
coerção, mas não impedindo-as.

Este protocolo explicitamente não lida com o problema da coerção, pois com a
natureza descentralizada do sistema proposto é excessivamente complexo
controlar cada terminal de votação. Em vez disso, a possibilidade de substituir
o voto a qualquer momento permite que o eleitor vote até mesmo em frente a um
agressor e posteriormente desfaça seu voto. Esta é a mesma abordagem do sistema
Helios.

Encerrada a coleta de votos, no quinto passo o sistema faz a totalização dos
votos, somando-os de forma homomórfica baseada nas propriedades do
criptosistema de Paillier. O sistema deve considerar somente os votos
corretamente formados e mais recentes de cada eleitor.

No sexto passo, as autoridades submetem suas decifragens parciais do somatório
dos votos, obtidos utilizando as chaves privadas parciais distribuídas no
terceiro passo, acompanhados de provas da decifragem correta. Pelo esquema de
\textcite{fouque2000sharing}, nenhuma autoridade sozinha pode obter o resultado
final da eleição, somente uma combinação de pelo menos $t$ decifragens parciais
permite encontrar o resultado total decifrado. Se menos de $t$ autoridades
enviarem resultados parciais válidos até o prazo expirar para este passo, a
eleição é encerrada com falha.

No sétimo e último passo, o sistema combina as decifragens parciais e gera a
totalização, que é então publicada no \textit{ledger} junto com os outros dados
da eleição, e encerra com sucesso.

O protocolo apresentado é uma simplificação do protocolo ADDER, uma vez que a
geração de chaves do criptosistema ElGamal utilizado no ADDER é dividido em
mais passos. A ideia de simplificar o protocolo e reduzir a necessidade de
interação também tem o objetivo de facilitar o uso por autoridades e eleitores
com menor conhecimento técnico.

\section{Modularidade}

Seguindo tendências modernas para o desenvolvimento de sistemas \textit{web}, o
\textit{software} que faz parte desta implementação é logicamente dividido
entre \textit{back-end}, composto pelo \textit{ledger} e os serviços que este
acessa (como os serviços de autenticação), e o \textit{front-end}, composto
pelos clientes responsáveis pela interação com o usuário. Desta forma, é
possível que cada parte tenha menos responsabilidade e possibilidades de falha
do que o tradicional método, onde uma única aplicação faz tanto a interação com
o usuário quanto a manipulação do banco de dados~\cite{lanthaler2012using}.

Outra vantagem desta abordagem é que desenvolvedores independentes podem
construir outros clientes, como aplicativos para celular ou até mesmo terminais
especializados para esquemas de eleição alternativos baseados nesta
implementação, sem a necessidade de modificar o \textit{back-end} para suportar
tais adaptações.

Esta tendência vem em conjunto com \textit{frameworks} para interfaces de
usuário como Angular, React e Vue, que transferem a responsabilidade de
apresentar as informações para o cliente, em vez de renderizar as páginas HTML
no servidor. Técnicas agressivas de \textit{cache} também favorecem, já que a
quantidade de informação transmitida entre o cliente e o servidor se limita ao
mínimo necessário para cada atividade e há maior reutilização de código quando
o conteúdo é construído dinamicamente~\cite{souders2008high}.

\subsection{Comunicação entre os módulos}

A comunicação entre o cliente e o servidor se dá através de uma linguagem de
representação de dados, sendo a mais comum atualmente o \textbf{JSON}. O canal
de comunicação precisa ser seguro e privado, portanto sempre ocorrendo por um
canal com protocolo SSL ou TLS, como o HTTPS~\cite{rfc2818} no caso de um
cliente de \textit{browser}.

Computadores e celulares são rotineiramente atacados por software malicioso,
como vírus e \textit{malwares}, e requerem cuidado especial da parte do
usuário, no caso o eleitor. A segurança neste lado não pode ser garantida pelo
sistema eleitoral e é, portanto, uma preocupação para a adoção de sistemas de
eleição remotos, mas novas tecnologias e padrões recentemente desenvolvidos e
implementados pelos navegadores permitem maiores níveis de segurança para
aplicações \textit{web}.

\subsubsection{Mitigação de vulnerabilidades}

Dentre as tecnologias desenvolvidas nos últimos anos para mitigar
vulnerabilidades no \textit{browser}, são as mais conhecidas:
\textit{Content Security Policy} (\textbf{CSP}), que visa controlar recursos
que uma aplicações \textit{web} pode executar~\cite{west2016csp};
\textit{HTTP Strict Transport Security} (\textbf{HSTS}), que protege de ataques
de \textit{downgrade} de protocolo, forçando acesso somente por canais
criptografados e seguros~\cite{rfc6797}; \textit{HTTP Public Key Pinning}
(\textbf{HPKP}), que associa chaves públicas com uma aplicação e evita riscos
de ataques com certificados forjados~\cite{rfc7469};
\textit{Subresource Integrity} (\textbf{SRI}), que provê formas do navegador
assegurar que recursos foram corretamente transferidos sem
manipulação~\cite{akhawe2016sri}; e as inúmeras recomendações da iniciativa
\textit{Open Web Application Security Project} (\textbf{OWASP}), uma
organização sem fins lucrativos que distribui documentação e ferramentas para
segurança em aplicações \textit{web}. A maioria destas soluções diz respeito ao
cliente, não ao servidor.

No servidor, além dos cuidados com infecção maliciosa, que poderiam ser
evitadas por profissionais técnicos, também é necessário lidar com outros tipos
de invasão e negação de serviço. Um ou mais \textit{insiders}, sejam eles
técnicos ou programadores responsáveis pelo sistema podem manipular o software
de forma indetectável, e as recentes acusações dos Estados Unidos sobre uma
suposta manipulação nas suas eleições por espiões da Rússia certamente não
melhora a opinião pública~\cite{badawy2018analyzing}.

Ataques de negação de serviço (\textbf{DoS}, na sigla para
\textit{Denial of Service}) ocorrem quando um agressor acessa um serviço com
uma grande quantidade de dispositivos simultaneamente, com o objetivo de
congestionar e sobrecarregar a rede e o servidor, impedindo que outros usuários
o acessem.  Ataques desse tipo em larga escala tipicamente utilizam milhares ou
milhões de computadores previamente infectados com vírus, dificultando sua
mitigação, pois não há uma origem consistente dos acessos. Este tipo de ataque
também pode ser utilizado para distrair equipes de segurança e explorar outras
falhas no sistema enquanto as defesas estão focadas em mitigar a sobrecarga.

Uma forma de ataque de negação de serviço conhecida como \textit{Slowloris},
que consiste em abrir uma quantidade de conexões com o servidor maior do que
ele pode suportar, mantendo-as abertas e impedindo que o servidor aceite novas
conexões de usuários legítimos, foi utilizada em sites do governo iraniano
durante as eleições presidenciais de 2009 após inúmeros
protestos~\cite{zdrnja2009slowloris}.


