\section{Cadastro e autenticação de eleitores}

O subsistema de cadastro e autenticação de eleitores é responsável por
registrar os dados e confirmar a identidade dos eleitores que acessem o sistema
de votação. Nos sistemas citados no \autoref{chap:trabalhos}, são utilizados
diferentes sistemas de registro e autenticação, notadamente o uso de
\textit{tokens} enviados para os eleitores antes da votação.

O sistema Helios tem suporte a \textit{plugins} de autenticação, suportando
padrões abertos de \textit{single sign-on} como CAS, OAuth, Shibboleth e
sistemas de identificação federados como o OpenID através destes. Pode também,
teoricamente, suportar autenticação através de \textit{smart cards} e outras
formas de certificados digitais, mas seu repositório não integra estas soluções
atualmente.

A terceirização do serviço de autenticação permite delegar a confiabilidade
deste para outra autoridade dedicada. Por exemplo, diversas universidades
possuem serviços de autenticação centralizada utilizando algum dos protocolos
citados anteriormente, e a própria entidade garante a autenticidade dos seus
registros e a autorização do seu uso. Isso reduz a superfície de ataque do
sistema de eleição e o torna mais flexível e adaptável a novas tecnologias.
