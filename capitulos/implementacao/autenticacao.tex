\section{Cadastro e autenticação de eleitores}

O subsistema de cadastro e autenticação de eleitores é responsável por
registrar os dados e confirmar a identidade dos eleitores que acessem o sistema
de votação. Nos sistemas citados no \autoref{ch:trabalhos correlatos}, são
utilizados diferentes sistemas de registro e autenticação, notadamente o uso de
\textit{tokens} enviados para os eleitores antes da votação.

O sistema Helios tem suporte a \textit{plugins} de autenticação, suportando
padrões abertos de \textit{single sign-on} como CAS, OAuth, Shibboleth e
sistemas de identificação federados como o OpenID Connect através destes. Pode
também, teoricamente, suportar autenticação através de \textit{smart cards} e
outras formas de certificados digitais, mas seu repositório não integra nenhuma
destas soluções de autenticação citadas atualmente.

A terceirização do serviço de autenticação permite delegar a confiabilidade
deste para outra autoridade dedicada. Por exemplo, diversas universidades
possuem serviços de autenticação centralizada utilizando algum dos protocolos
citados anteriormente, e a própria entidade garante a autenticidade dos seus
registros e a autorização do seu uso. Isso reduz a superfície de ataque do
sistema de eleição e o torna mais flexível e adaptável a novas tecnologias.

Para a implementação de referência desenvolvida neste trabalho, optou-se por
utilizar o protocolo OpenID Connect~\cite{sakimura2014openid}, utilizando como
servidor de autorização o Google. Desta forma, a autenticação de usuários é
realizada por um terceiro, e o servidor do registro de votos não gerencia os
usuários, se limitando a armazenar apenas o nome e e-mail.

Utiliza-se também a tecnologia \textit{Credential Management API}
(\textbf{CM-API}) para armazenar e recuperar credenciais federadas diretamente
no \textit{browser}, simplificando o fluxo de reautenticação de uma sessão
expirada~\cite{Credenti82:online}. Essa tecnologia ainda é considerada
experimental, estando disponível somente no navegador Chrome no momento de
escrita deste trabalho.

A combinação destas duas tecnologias permite armazenar o mínimo possível de
informação sobre o eleitor no servidor de eleição, delegando a tarefa de
autenticação para outros serviços de confiança do próprio usuário, como o
navegador de sua preferência e o serviço de autenticação a ser consultado.

A \textbf{CM-API} também prevê suporte para autenticação por senha e por chave
pública. Para a autenticação por chave pública, é possível utilizar
dispositivos de chave assimétrica, como \textit{smart cards} contendo
certificados digitais para autenticação física dos usuários, se este nível de
segurança for desejado. Nenhum navegador suporta este método no momento da
escrita deste trabalho, mas Chrome e Firefox já possuem implementações
parciais.
