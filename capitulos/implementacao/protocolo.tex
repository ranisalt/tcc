\section{Protocolo de uma eleição}

O protocolo de uma eleição utilizado no ADDER, e portanto no qual é inspirado
este trabalho, se concentra em torno do registro digital público de votos,
chamado de \textit{bulletin board} ou \textit{ledger}, e está descrito em
\textcite{benaloh1987verifiable}. Este registro consiste em um mural público
onde todos os usuários antenticados, tanto eleitores, autoridades e
administradores podem inserir dados.

Na implementação, o \textit{ledger} consiste em um banco de dados relacional
acessado através de uma interface \textit{web} que armazena as credenciais de
uma eleição e versões criptografadas dos votos, e um servidor de autenticação
que emite \textit{tokens} com os quais os usuários podem se autenticar com o
\textit{ledger} e realizar ações de acordo com sua permissão. Uma interface
\textit{web} acessa os dados destes servidores de forma transparente.

O procedimento de execução de uma eleição é dividido em passos sequenciais para
criar as informações necessárias, receber os votos e calcular o resultado
final.

Para criar uma nova eleição, um administrador do sistema envia os parâmetros da
nova eleição para o \textit{ledger} através da interface \textit{web}. Estes
parâmetros são os identificadores da eleição, a lista de $n$ autoridades, a
lista de eleitores, a quantidade mínima $m_{min}$ e máxima $m_{max}$ de
candidatos por voto, a lista de candidatos, o limite mínimo de autoridades $t$
e os horários limites para os passos seguintes. Estas informações ficam
publicamente disponíveis no \textit{ledger} para conferência.

No segundo passo, as autoridades devem submeter suas chaves públicas para o
sistema, para que toda a comunicação de informação sensível entre o sistema e a
autoridade se dê por forma criptografada, nunca em texto claro. A autoritade
também deve submeter um desafio para provar que detém a chave privada
relacionada à chave pública enviada. Se menos de $t$ autoridades enviarem
chaves públicas até o prazo expirar para este passo, a eleição é encerrada com
falha.

No terceiro passo, o administrador que criou a eleição solicita que o sistema
gere o par de chaves da eleição. Para isso, utiliza o algoritmo de geração de
chaves descrito em \textcite{fouque2000sharing}, que cria uma chave pública,
divide a chave privada entre as autoridades que participaram do passo anterior
e envia cada parte por e-mail de forma criptografada. A chave pública e as
chaves de verificação são publicados no \textit{ledger} junto com os outros
detalhes da eleição gerados até agora. Se menos de $t$ autoridades confirmarem
o recebimento de suas chaves privadas parciais até o prazo expirar para este
passo, a eleição é encerrada com falha.

Antes do quarto passo, que é a votação de fato, o administrador deve publicar a
lista de eleitores no \textit{ledger}. Os eleitores são identificados
unicamente por e-mail, e como especificado anteriormente, são autenticados por
um servidor de autenticação externo.

Durante a votação, somente os eleitores submetidos anteriormente podem inserir
seus votos no \textit{ledger}. Os votos são cifrados e acompanhados de uma
prova de formação correta, de forma que o eleitor pode conferir se seu voto
cifrado está corretamente publicado e o sistema pode garantir que o eleitor
votou em pelo menos $m_{min}$ opções e em até $m_{max}$ opções. O eleitor pode
substituir seu voto quantas vezes desejar, de forma a mitigar tentativas de
coerção, mas não impedindo-as.

Este protocolo explicitamente não lida com o problema da coerção, pois com a
natureza descentralizada do sistema proposto é excessivamente complexo
controlar cada terminal de votação. Em vez disso, a possibilidade de substituir
o voto a qualquer momento permite que o eleitor vote até mesmo em frente a um
agressor e posteriormente desfaça seu voto. Esta é a mesma abordagem do sistema
Helios.

Encerrada a coleta de votos, no quinto passo o sistema faz a totalização dos
votos, somando-os de forma homomórfica baseada nas propriedades do
criptosistema de Paillier. O sistema deve considerar somente os votos
corretamente formados e mais recentes de cada eleitor.

No sexto passo, as autoridades submetem suas decifragens parciais do somatório
dos votos, obtidos utilizando as chaves privadas parciais distribuídas no
terceiro passo, acompanhados de provas da decifragem correta. Pelo esquema de
\textcite{fouque2000sharing}, nenhuma autoridade sozinha pode obter o resultado
final da eleição, somente uma combinação de pelo menos $t$ decifragens parciais
permite encontrar o resultado total decifrado. Se menos de $t$ autoridades
enviarem resultados parciais válidos até o prazo expirar para este passo, a
eleição é encerrada com falha.

No sétimo e último passo, o sistema combina as decifragens parciais e gera a
totalização, que é então publicada no \textit{ledger} junto com os outros dados
da eleição, e encerra com sucesso.

O protocolo apresentado é uma simplificação do protocolo ADDER, uma vez que a
geração de chaves do criptosistema ElGamal utilizado no ADDER é dividido em
mais passos. A ideia de simplificar o protocolo e reduzir a necessidade de
interação também tem o objetivo de facilitar o uso por autoridades e eleitores
com menor conhecimento técnico.
