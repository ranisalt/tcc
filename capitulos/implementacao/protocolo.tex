\section{Protocolo de uma eleição}

O protocolo de uma eleição utilizado no ADDER, e portanto no qual é inspirado
este trabalho, se concentra em torno do registro digital público de votos,
chamado de \textit{bulletin board} ou \textit{ledger}, e está descrito em
\textcite{benaloh1987verifiable}. Este registro consiste em um mural público
onde todos os usuários antenticados, tanto eleitores, autoridades e
administradores podem inserir dados.

Na implementação, o \textit{ledger} consiste em um banco de dados relacional
acessado através de uma interface \textit{web} que armazena as credenciais de
uma eleição e versões criptografadas dos votos, e um servidor de autenticação
que emite \textit{tokens} com os quais os usuários podem se autenticar com o
\textit{ledger} e realizar ações de acordo com sua permissão. Uma interface
\textit{web} acessa os dados destes servidores de forma transparente.

O procedimento de execução de uma eleição é dividido em passos sequenciais para
criar as informações necessárias, receber os votos e calcular o resultado
final.

Para criar uma nova eleição, um administrador do sistema envia os parâmetros da
nova eleição para o \textit{ledger} através da interface \textit{web}. Estes
parâmetros são os identificadores da eleição, a lista de autoridades, a lista
de eleitores, a quantidade mínima e máxima de candidatos por voto, a lista de
candidatos, o limite mínimo de autoridades $t$ e os horários limites para os
passos seguintes. Estas informações ficam publicamente disponíveis no
\textit{ledger} para conferência.
