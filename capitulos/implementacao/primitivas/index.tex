\section{Biblioteca de primitivas criptográficas}

Como esta implementação utiliza rotinas criptográficas genéricas, é possível
criar uma biblioteca dedicada a estas primitivas, para ser usada não apenas
nesta implementação, mas em qualquer \textit{software} que deseje utilizá-las.
Dessa forma, o código pode ser testado mais vigorosamente por aplicações de
diferentes finalidades, aumentando a confiabilidade e a probabilidade de que
uma falha seja detectada~\cite{hursch1995separation}.

Esta biblioteca é responsável pela geração dos números primos necessários no
criptosistema utilizado, utilidades de verificação de primalidade, geração de
pares de chaves e compartilhamento de segredos, que serão descritos nas
próximas subseções.

A biblioteca foi denominada \textbf{swartz} em homenagem a Aaron Swartz
(1986-2013), um programador e ativista que foi preso ao ser acusado de roubar
documentos de um serviço de periódicos online para distribuir ilegalmente,
acusação esta que nunca foi provada~\cite{sims2011library}.

\section{Criptosistema de Paillier}

Para realizar a criptografia dos votos para armazenamento, é necessário um
sistema criptográfico de chaves assimétricas em que o subsistema de registro de
votos possa realizar a contagem sem conhecimento da chave privada para
descriptografar os dados. Isso implica que o sistema criptográfico utilizado
tenha propriedades homomórficas, para que seja possível realizar operações com
resultado conhecido diretamente no texto cifrado.

A propriedade homomórfica é utilizada na contagem para que o resultado
criptografado possa ser publicado para então ser aberto pelas autoridades
detentoras da chave privada. Deve ser possível gerar o somatório dos votos
apenas com a cifragem de cada voto.

O criptosistema de Paillier~\cite{paillier1999public} foi escolhido por possuir
estas características, sendo um criptosistema de chaves assimétricas com
propriedades homomórficas cuja complexidade é dada pela hipótese de
intratabilidade da residuosidade composta de decisão (do inglês
\textit{decisional composite residuosity}, \textbf{DCR}).

Além disso, o criptosistema de Paillier utiliza uma chave efêmera aleatória
para cifragem, o que gera uma mensagem cifrada diferente cada vez que a mesma
mensagem original é cifrada. Portanto, mesmo se dois eleitores votarem
exatamente nas mesmas opções, o voto cifrado armazenado no banco de dados será
diferente para cada um.

Essa propriedade permite que os votos de um eleitor sejam representados pelos
valores $0$ e $1$ caso o eleitor vote contra ou a favor de uma opção,
respectivamente. Se a mensagem cifrada fosse igual para todos os valores
iguais, seria fácil observar o registro público do \autoref{sect:registro} dos
votos e obter informações sobre os votos sem conhecer a chave privada da
eleição.

A chave efêmera não é armazenada e não é necessária para decifrar a mensagem,
portanto não é possível reproduzir o voto somente com as informações
armazenadas no registro a fim de quebrar o sigilo do voto.

\subsection{Compartilhamento de segredos de Shamir}

Para dividir a responsabilidade de revelar a contagem final de votos entre
múltiplas autoridades, se faz necessário um esquema de compartilhamento de
segredos. Em \textcite{Shamir:1979:SS:359168.359176} é descrito um esquema de
compartilhamentos baseado em polinômios, cuja ideia central é de que são
necessários $k$ pontos para se definir um polinômio de grau $k - 1$.

Utilizando este esquema, definimos um polinômio de grau $k - 1$, onde $k$ é a
quantidade mínima de autoridades necessárias para se revelar os textos
cifrados, dentre um total de $n$ autoridades tal que $0 < k \leq n$. Cada
autoridade calcula sua decifragem parcial $c_i$ que será utilizada para
reconstruir a mensagem original $m$ quando combinadas ao menos $k$ decifragens
parciais.

A definição do criptosistema de Paillier modificado para suportar o
compartilhamento de segredos de Shamir utilizado neste trabalho encontra-se em
\textcite{fouque2000sharing}. Embora haja uma implementação mais genérica em
\textcite{damgaard2010generalization}, esta possui restrições quanto aos
valores de $k$ em relação a $n$ que foram consideradas indesejáveis.

