\subsection{Criptosistema de Paillier}

Para realizar a criptografia dos votos para armazenamento, é necessário um
sistema criptográfico de chaves assimétricas em que o subsistema de registro de
votos possa realizar a contagem sem conhecimento da chave privada para
descriptografar os dados. Isso implica que o sistema criptográfico utilizado
tenha propriedades homomórficas, para que seja possível realizar operações com
resultado conhecido diretamente no texto cifrado.

A propriedade homomórfica é utilizada na contagem para que o resultado
criptografado possa ser publicado para então ser aberto pelas autoridades
detentoras da chave privada. Deve ser possível gerar o somatório dos votos
apenas com a cifragem de cada voto.

O criptosistema de Paillier~\cite{paillier1999public} foi escolhido por possuir
estas características, sendo um criptosistema de chaves assimétricas com
propriedades homomórficas cuja complexidade é dada pela hipótese de
intratabilidade da residuosidade composta de decisão (do inglês
\textit{decisional composite residuosity}, \textbf{DCR}).

Embora o ADDER originalmente utilize o criptosistema de ElGamal, durante este
trabalho o autor considerou o criptosistema de Paillier mais simples e com
pesquisas mais aprofundadas sobre sua segurança no âmbito de eleições online. É
plenamente possível que posteriormente seja implementado o criptosistema de
ElGamal como alternativa.

Além disso, o criptosistema de Paillier utiliza uma chave efêmera aleatória
para cifragem, o que gera uma mensagem cifrada diferente cada vez que a mesma
mensagem original é cifrada. Portanto, mesmo se dois eleitores votarem
exatamente nas mesmas opções, o voto cifrado armazenado no banco de dados será
diferente para cada um.

Essa propriedade permite que os votos de um eleitor sejam representados pelos
valores $0$ e $1$ caso o eleitor vote contra ou a favor de uma opção,
respectivamente. Se a mensagem cifrada fosse igual para todos os valores
iguais, seria fácil observar o registro público do \autoref{sect:registro} dos
votos e obter informações sobre os votos sem conhecer a chave privada da
eleição.

A chave efêmera não é armazenada e não é necessária para decifrar a mensagem,
portanto não é possível reproduzir o voto somente com as informações
armazenadas no registro a fim de quebrar o sigilo do voto.
