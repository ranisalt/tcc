\section{Registro público de votos}
\label{sect:registro}

O registro de votos é o subsistema responsável por receber e armazenar com
segurança os votos dos eleitores. Este sistema não deve ter capacidade de abrir
os votos depositados, não tendo acesso à chave privada utilizada para
decifrá-los. Dessa forma, sua responsabilidade é evitar votos duplicados,
permitindo que um eleitor sobrescreva seu voto, e reportar para o sistema de
contagem o resultado final.

O subsistema de registro não deve ter acesso à chave privada de nenhuma
eleição, para garantir o sigilo do voto. Deve somente fornecer uma interface
para a inserção e conferência dos votos, como faz uma urna, repassando o
somatório cifrado às autoridades de contagem ao final do processo.

Pode ser implementado como uma camada fina de software sobre um banco de dados
que confere a autenticação do usuário com o subsistema de autenticação,
armazena os votos válidos de forma cifrada neste banco de dados e provê uma
forma de visualização dos votos já inseridos, sem informações sobre os
eleitores.
