\section{Modularidade}

Seguindo tendências modernas para o desenvolvimento de sistemas \textit{web}, o
\textit{software} que faz parte desta implementação é logicamente dividido
entre \textit{back-end}, composto pelo \textit{ledger} e os serviços que este
acessa (como os serviços de autenticação), e o \textit{front-end}, composto
pelos clientes responsáveis pela interação com o usuário. Desta forma, é
possível que cada parte tenha menos responsabilidade e possibilidades de falha
do que o tradicional método, onde uma única aplicação faz tanto a interação com
o usuário quanto a manipulação do banco de dados.

Outra vantagem desta abordagem é que desenvolvedores independentes podem
construir outros clientes, como aplicativos para celular ou até mesmo terminais
especializados para esquemas de eleição alternativos baseados nesta
implementação, sem a necessidade de modificar o \textit{back-end} para suportar
tais adaptações.

Esta tendência vem em conjunto com \textit{frameworks} para interfaces de
usuário como Angular, React e Vue, que transferem a responsabilidade de
apresentar as informações para o cliente, em vez de renderizar as páginas HTML
no servidor. Técnicas agressivas de \textit{cache} também favorecem, já que a
quantidade de informação transmitida entre o cliente e o servidor se limita ao
mínimo necessário para cada atividade e há maior reutilização de código quando
o conteúdo é construído dinamicamente~\cite{souders2008high}.

\subsection{Comunicação entre os módulos}

A comunicação entre o cliente e o servidor se dá através de uma linguagem de
representação de dados, sendo a mais comum atualmente o \textbf{JSON}. O canal
de comunicação precisa ser seguro e privado, portanto sempre ocorrendo por um
canal com protocolo SSL ou TLS, como o HTTPS~\cite{rfc2818} no caso de um
cliente de \textit{browser}.

Computadores e celulares são rotineiramente atacados por software malicioso,
como vírus e \textit{malwares}, e requerem cuidado especial da parte do
usuário, no caso o eleitor. A segurança neste lado não pode ser garantida pelo
sistema eleitoral e é, portanto, um empecilho para a adoção de sistemas de
eleição pela internet, mas novas tecnologias e padrões recentemente
desenvolvidos e implementados pelos navegadores permitem maiores níveis de
segurança para aplicações \textit{web}.

\subsubsection{Mitigação de vulnerabilidades}

Dentre as tecnologias desenvolvidas nos últimos anos para mitigar
vulnerabilidades no \textit{browser}, são as mais conhecidas:
\textit{Content Security Policy} (\textbf{CSP}), que visa controlar recursos
que uma aplicações \textit{web} pode executar~\cite{west2016csp};
\textit{HTTP Strict Transport Security} (\textbf{HSTS}), que protege de ataques
de \textit{downgrade} de protocolo, forçando acesso somente por canais
criptografados e seguros~\cite{rfc6797}; \textit{HTTP Public Key Pinning}
(\textbf{HPKP}), que associa chaves públicas com uma aplicação e evita riscos
de ataques com certificados forjados~\cite{rfc7469};
\textit{Subresource Integrity} (\textbf{SRI}), que provê formas do navegador
assegurar que recursos foram corretamente transferidos sem
manipulação~\cite{akhawe2016sri}; e as inúmeras recomendações da iniciativa
\textit{Open Web Application Security Project} (\textbf{OWASP}), uma
organização sem fins lucrativos que distribui documentação e ferramentas para
segurança em aplicações \textit{web}. A maioria destas soluções diz respeito ao
cliente, não ao servidor.

No servidor, além dos cuidados com infecção maliciosa, que poderiam seriam
evitadas por profissionais técnicos, também é necessário lidar com outros tipos
de invasão e negação de serviço. Um ou mais \textit{insiders}, sejam eles
técnicos ou programadores responsáveis pelo sistema podem manipular o software
de forma indetectável, e as recentes acusações dos Estados Unidos sobre uma
suposta manipulação nas suas eleições por espiões da Rússia certamente não
melhora a opinião pública~\cite{badawy2018analyzing}.

Ataques de negação de serviço (\textbf{DoS}, na sigla para
\textit{Denial of Service}) ocorrem quando um agressor acessa um serviço com
uma grande quantidade de dispositivos simultaneamente, com o objetivo de
congestionar e sobrecarregar a rede e o servidor, impedindo que outros usuários
o acessem.  Ataques desse tipo em larga escala tipicamente utilizam milhares ou
milhões de computadores previamente infectados com vírus, dificultando sua
mitigação, pois não há uma origem consistente dos acessos. Este tipo de ataque
também pode ser utilizado para distrair equipes de segurança e explorar outras
falhas no sistema enquanto as defesas estão focadas em mitigar a sobrecarga.

Uma forma de ataque de negação de serviço conhecida como \textit{Slowloris},
que consistem em abrir uma quantidade de conexões com o servidor maior do que
ele pode suportar, mantendo-as abertas e impedindo que o servidor aceite novas
conexões de usuários legítimos, foi utilizada em sites do governo iraniano
durante as eleições presidenciais de 2009 após inúmeros
protestos~\cite{zdrnja2009slowloris}.

