\section{Civitas}

O sistema Civitas, descrito em \textcite{clarkson2008civitas}, tem suas raízes no \textit{Condorcet Internet Voting System} (CIVS), um sistema pesquisa online e aberto que permite qualquer usuário acesse o sistema e crie uma eleição.

O Civitas é desenhado para resistir a coerção, permitindo que um eleitor ameaçado utilize credenciais falsas para votar sem modificação aparente no sistema, que será descartado na apuração, baseado em protocolo descrito por \textcite{juels2005coercion}.

O eleitor também não pode se corromper propositalmente, pois não há prova de voto mesmo em caso de interação com o agressor no momento do voto, pois pode permitir votos múltiplos considerando o voto final no resultado.

O sistema é compatível com diferentes modelos de cédulas, como cédulas de voto em candidato único, voto em conjunto de candidatos, e voto por classificação (opções ordenadas por preferência).

O Civitas utiliza um sistema criptográfico distribuído com resistência a falhas, sendo uma falha definida por um agente responsável (autoridade) pela eleição desvie do protocolo padrão, seja por comportamento malicioso, erros não intencionais ou simplesmente pela não colaboração~\cite{davis2008civitas}. Para estas situações, há uma quantidade mínima configurável de agentes necessários para realizar a apuração menor ou igual à quantidade total de agentes responsáveis.