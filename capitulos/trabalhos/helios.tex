\section{Helios}

O sistema Helios, descrito em \textcite{adida2008helios}, utiliza um protocolo
que possui similaridades com ADDER. Foi desenvolvido como uma plataforma de
eleição online, e suporta integração com diversos sistemas de \textit{login},
como CAS e OAuth.

A construção da cédula é baseada no protocolo descrito por
\textcite{benaloh2006simple}, que separa os processos de preparação e envio da
cédula, de forma que mesmo um participante que não seja apto para votar possa
acompanhar o processo de preparação, e apenas no passo final da cabine de
votação a autenticação do eleitor se faz necessária.

De um ponto de vista técnico, a descentralização da verificação do Helios
permite alta confiabilidade no processo, uma vez que não há um ponto central de
falha na votação e cada participante pode confirmar a contabilização do seu
voto.

No entanto, o sistema sofre com problemas de
usabilidade~\cite{karayumak2011usability}, já que os eleitores não possuem
prática suficiente para tal procedimento, muitas vezes ignorando a auditoria,
que é uma das vantagens de sistemas auditáveis fim-a-fim.

O Helios tem um funcionamento mais simples para o leitor que o Civitas, pois
este último é dividido em subsistemas separados com única funcionalidade que
trocam o mínimo necessário de informações, enquanto o primeiro é mais próximo
de um sistema monolítico.
