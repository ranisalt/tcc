\chapter*{Lista de definições}

\begin{description}[style=nextline]
	\item[Auditabilidade fim-a-fim] Característica de sistemas de eleição com forte integridade e resistência à violação. Como o nome sugere, um sistema com tal característica pode ser verificado desde o voto até o resultado final pela sua corretude~\cite{benaloh2015end}.

	\item[Autenticação] Possibilidade da autoridade eleitoral de reconhecer o eleitor e confirmar o seu direito ao voto. Pode ser utilizada uma lista de nomes, um certificado de sociedade, ou um documento eleitoral.
    
    \item[Criptografia homomórfica] Forma de criptografia que permite operações no texto cifrado, gerando um resultado que, quando descriptografado, seja equivalente a aplicar as mesmas operações no texto original~\cite{rivest2002lecture}. Sua principal aplicação é manipular informação em ambientes não confiáveis.

    \item[Integridade eleitoral] Conjunto de ações para que o resultado das eleições seja de fato a representação do interesse dos eleitores. As entidades do sistema devem contabilizar corretamente cada voto de acordo com a intenção do eleitor, e o resultado final deve ser exatamente a soma destes votos. Além disso, deve haver liberdade para o registro do voto de qualquer eleitor habilitado~\cite{alvim2015integridade}.
    
    \item[Mentalidade de segurança] Forma de pensar em soluções e abordar problemas explorando vulnerabilidades de segurança, ou seja, analisar o cenário do ponto de vista de um agressor de forma a detectar falhas antes de serem potencialmente exploradas~\cite{schneier2008inside}.
    
    \item[Voto privado] Proteção do eleitor de que seu ato de votar não poderá ser observado por outros participantes do processo. É uma forma de proteção mais fraca que o voto secreto, mas é suficiente para alguns cenários.
    
    \item[Voto secreto] Impossibilidade do sistema de votação fazer qualquer ligação entre votante e voto, impedindo de revelar em que candidato(s) um eleitor votou~\cite{delaune2006coercion}. Para garantias mais fortes, pode ser desejável que nem mesmo o eleitor possa revelar em quem votou.
\end{description}

\chapter{Introdução}

Mesmo com a ubiquidade da internet atualmente, onde fazemos desde compras domésticas até transações bancárias com segurança, a eleição remota é atualmente um tema bastante controverso no mundo da tecnologia da informação. No Brasil, a resistência do tribunal eleitoral de conduzir auditorias irrestritas do sistema eletrônico, acompanhada das vulnerabilidades descobertas no sistema eletrônico, mesmo nos cenários restritos onde auditorias foram realizadas~\cite{aranhavulnerabilidades}, cria na população uma sensação de inexistência de integridade nas eleições.

O uso de um meio rápido e descentralizado como a internet pode trazer diversos benefícios para o eleitor, com destaque para a comodidade de exercer seu voto de qualquer lugar, a redução do tempo de contagem e transporte dos sistemas clássicos de cédulas ou urnas eletrônicas. Por isso, existem tentativas de se utilizar sistemas de votação remota, como nas eleições nacionais da Estônia. No entanto, estas estão aquém dos requisitos de segurança esperados~\cite{Springall:2014:SAE:2660267.2660315}, havendo incerteza se é segura o suficiente para a realização de eleições.

Dos diversos sistemas eletrônicos existentes ou em desenvolvimento, os mais populares não estão disponíveis publicamente para uso e estudo, não atendem aos requisitos de segurança de uma eleição, e não passaram por uma extensa bateria de testes de segurança. É consenso na comunidade de segurança digital que segurança por obscurantismo não é eficiente, e se a eleição acontece em uma "caixa preta" não existem meios de verificar a validade dos resultados.

Este trabalho visa estudar e implementar uma solução de eleição remota baseada no sistema ADDER~\cite{kiayias2006internet}. No seu artigo, o autor citou trabalhos futuros a serem desenvolvidos no sistema que serão abordados no presente trabalho.

\section{Justificativa}

Em virtude dos ataques de que exploram falhas de segurança, como os \textit{ransomware} presenciados nos anos recentes, observando a falta de conhecimento da população sobre segurança digital, e pensando em uma solução para um dos grandes problemas políticos do país, este trabalho visa apresentar os recentes avanços de segurança digital em uma proposta de sistema seguro que possa ser utilizado para tal.

De forma mais geral, qualquer votação altamente descentralizada e com uma grande quantidade de eleitores participantes, os sistemas mais ortodoxos e amplamente testados sofrem com problemas de escalabilidade. Considerando que a internet é o meio de comunicação e interação mais ubíquo, a digitalização de sistemas de votação é de grande importância tanto pela acessibilidade permitida, quanto pela segurança provida pelas modernas técnicas de criptografia.