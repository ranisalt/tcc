\chapter*{Lista de definições}

\begin{description}[style=nextline]
	\item[Auditabilidade fim-a-fim] Característica de sistemas de eleição com forte integridade e resistência à violação. Como o nome sugere, um sistema com tal característica pode ser verificado desde o voto até o resultado final pela sua corretude~\cite{benaloh2015end}.

	\item[Autenticação] Possibilidade da autoridade eleitoral de reconhecer o eleitor e confirmar o seu direito ao voto. Pode ser utilizada uma lista de nomes, um certificado de sociedade, ou um documento eleitoral.
    
    \item[Criptografia homomórfica] Forma de criptografia que permite operações no texto cifrado, gerando um resultado que, quando descriptografado, seja equivalente a aplicar as mesmas operações no texto original~\cite{rivest2002lecture}. Sua principal aplicação é manipular informação em ambientes não confiáveis.

    \item[Integridade eleitoral] Conjunto de ações para que o resultado das eleições seja de fato a representação do interesse dos eleitores. As entidades do sistema devem contabilizar corretamente cada voto de acordo com a intenção do eleitor, e o resultado final deve ser exatamente a soma destes votos. Além disso, deve haver liberdade para o registro do voto de qualquer eleitor habilitado~\cite{alvim2015integridade}.
    
    \item[Mentalidade de segurança] Forma de pensar em soluções e abordar problemas explorando vulnerabilidades de segurança, ou seja, analisar o cenário do ponto de vista de um agressor de forma a detectar falhas antes de serem potencialmente exploradas~\cite{schneier2008inside}.
    
    \item[Voto privado] Proteção do eleitor de que seu ato de votar não poderá ser observado por outros participantes do processo. É uma forma de proteção mais fraca que o voto secreto, mas é suficiente para alguns cenários.
    
    \item[Voto secreto] Impossibilidade do sistema de votação fazer qualquer ligação entre votante e voto, impedindo de revelar em que candidato(s) um eleitor votou~\cite{delaune2006coercion}. Para garantias mais fortes, pode ser desejável que nem mesmo o eleitor possa revelar em quem votou.
\end{description}