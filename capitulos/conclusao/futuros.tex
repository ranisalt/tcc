\section{Trabalhos futuros}

Um assunto que não foi abordado neste trabalho foi o aspecto legal de uma
solução descentralizada como a proposta, ou como categorizar o dispositivo
utilizado pelo eleitor em uma eleição descentralizada. Caso estes dispositivos
sejam considerados parte do equipamento eleitoral, e portanto sujeitos à lei
eleitoral, que criminaliza tentativas de quebra de sigilo e limita propaganda
eleitoral, seria praticamente impossível fiscalizar dado que milhares de
dispositivos estão em uso, portanto essa mudança de paradigma exigiria mudanças
na lei eleitoral caso a implementação seja usada numa eleição governamental.

O uso de uma infraestrutura de chaves públicas (\textbf{ICP}) para
gerenciamento das chaves das autoridades no segundo passo do protocolo da
eleição terceirizaria essa responsabilidade para um diretório dedicado,
reduziria ainda mais as responsabilidades do sistema implementado e poderia
tornar este passo totalmente automático, caso todas as autoridades já tivessem
chaves públicas inseridas em uma ICP. Esta modificação no protocolo também
aumenta o grau de confiança na identidade da autoridade, pois a tarefa de
submeter a chave pública é feita por um terceiro, e não pela própria
autoridade~\cite{nash2001pki}.

Também seria interessante avaliar formas de remover a necessidade da entidade
do administrador do sistema de uma eleição, embora este já possua
responsabilidade reduzida, tornando o processo da eleição dependente somente
das autoridades e eleitores. Esta é mais uma mudança de paradigma, pois
eleições historicamente são gerenciadas por uma entidade dedicada, como o
Tribunal Superior Eleitoral no caso das eleições governamentais, e demais
comissões eleitorais.

Uma tecnologia recente com potencial para melhorar a integridade do processo de
uma eleição são \textit{blockchains}, registros de transações
criptograficamente ligados em sequência. \textit{Blockchains} são fundamentadas
por consenso descentralizado e poderiam ser usadas para garantir que o sistema
não esteja modificando os votos de forma maliciosa ou por simples
mal-funcionamento~\cite{christidis2016blockchains}. Uma preocupação que deve
ser tomada ao se introduzir uma \textit{blockchain} num sistema eleitoral é de
prevenir o ordenamento dos votos, de forma a evitar a identificação do voto de
um eleitor específico.

Por fim, a implementação proposta precisa de uma verificação formal de seus
aspectos de segurança antes de ser utilizada em uma eleição de grande escala,
onde o nível de segurança proposto se faz necessário, como numa eleição
governamental ou de uma grande empresa. Embora não tenha sido proposto nenhum
novo algoritmo, é importante verificar a coesão de todos os procedimentos
utilizados.
