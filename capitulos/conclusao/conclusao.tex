Neste trabalho, foi desenvolvida uma implementação de sistema de eleição online
utilizando um esquema de criptografia homomórfica, com compartilhamento da
chave privada entre as autoridades da eleição, e cuja interface com o usuário é
desacoplada do sistema de armazenamento de dados, a fim de reduzir a quantidade
de informação trocada e, com isso, diminuir a superfície de ataques.

Este sistema é fundamentado em um protocolo robusto, porém simplificado para
comportar eleições gerenciadas por autoridades e utilizadas por eleitores com
menor conhecimento técnico sobre as primitivas criptográficas utilizadas.

Embora devamos tomar cuidado ao desenvolvermos uma alternativa remota para um
novo sistema eleitoral, é bastante satisfatório observar a evolução da
segurança computacional no passado recente. Enquanto eleições tem sido
aplicadas durante muitos séculos, e computadores só tenham sido inventados num
passado relativamente recente, eles tem rapidamente se tornado potenciais
alternativas para os meios de diversas atividades, entre elas as eleições.

O uso de sistemas criptográficos pode reforçar em ordens de magnitude a
segurança e privacidade das eleições, de forma muito mais eficiente e com menos
interferência humana do que os sistemas rudimentares de cédulas de papel, por
exemplo. Isso certamente tornará eleições mais íntegras e protegidas quando
estas premissas forem devidamente aplicadas.

As ferramentas necessárias para o desenvolvimento de sistemas eleitorais
remotos estão em pleno desenvolvimento, como demonstrado nas novas tecnologias
de segurança para navegadores e sistemas criptográficos modernos previamente
citados. Estas ferramentas permitirão novos níveis de segurança, mais próximos
do necessário para eleições.

Analisando o histórico da humanidade e da democracia, podemos esperar que no
futuro as eleições serão realizadas de forma descentralizada, e um sistema como
o proposto forma uma sólida base sobre a qual as eleições em qualquer instância
podem se tornar mais seguras, íntegras e práticas, considerando a intensa
penetração dos dispositivos móveis e informação de todos os serviços públicos.

O objetivo de construir uma solução com alto nível de segurança e privacidade
foi atingido, considerando a utilização do protocolo ADDER como base, e a
segurança provida pelo criptosistema de Paillier. Este criptosistema utiliza
primitivas criptográficas bem estabelecidas, e sua segurança é baseada na
dificuldade da fatoração de inteiros, que também provê a segurança do
criptosistema RSA, amplamente utilizado.

É importante salientar que as operações realizadas pela implementação são
bastante limitadas, de forma que o sistema final realiza somente a tarefa de
gerenciar a eleição, e explicitamente delega a tarefa de autenticação para um
terceiro, a fim de minimizar a responsabilidade e melhorar a extensibilidade do
sistema.
