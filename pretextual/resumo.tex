\begin{resumo}
    Este trabalho visa estudar sistemas de eleição através da internet, em
    especial o protocolo ADDER~\cite{kiayias2006internet}, abordando o
    histórico dos processos eleitorais, e propor uma implementação do processo
    eleitoral através de um \textit{website}. A implementação resultante
    combina o conceito de um registro público de
    informações~\cite{benaloh1987verifiable}, utilizando o criptosistema de
    Paillier~\cite{paillier1999public} para prover criptografia dos votos
    publicados e a divisão de responsabilidade entre
    autoridades~\cite{fouque2000sharing} a fim de promover um ambiente seguro
    que evita pontos únicos de falha para uma eleição online.

    \textbf{Palavras-chave}: criptografia. democracia. eleições. privacidade.
\end{resumo}

\begin{resumo}[Abstract]
    \begin{otherlanguage*}{english}
        The presented work aims to study internet-based election systems,
        notably the ADDER protocol~\cite{kiayias2006internet}, traversing the
        history of election processes, and to propose an implementation of an
        election process through a website. The resulting implementation
        combines the concept of a bulletin board~\cite{benaloh1987verifiable},
        while using the Paillier cryptosystem~\cite{paillier1999public} to
        publish encrypted votes and a threshold authority
        responsibility~\cite{fouque2000sharing} to provide a secure environment
        that avoids a single point of failure for an online election.

        \textbf{Keywords}: cryptography. democracy. elections. privacy.
    \end{otherlanguage*}
\end{resumo}
