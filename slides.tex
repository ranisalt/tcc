\documentclass[
    14pt, % base font size
    % aspectratio=169, % aspect ratio
]{beamer}
\usetheme[
    numbering=none,
    progressbar=frametitle,
    sectionpage=none,
]{metropolis}

\usepackage{pgfpages}
\setbeameroption{show notes on second screen}

\usepackage{csquotes,polyglossia}
\setdefaultlanguage{brazil}

\usepackage[style=abnt, language=brazil]{biblatex}
\addbibresource{biblio.bib}

\title{Implementação de um sistema de eleição remoto secreto e verificável}
\author{Ranieri Schroeder Althoff \\ Orientadora: Taciane Martimiano}
\date{\today}
\institute{%
    Universidade Federal de Santa Catarina -- UFSC
    \par
    Departamento de Informática e Estatística -- INE
    \par
    Programa de Graduação em Ciências da Computação}

\begin{document}

\maketitle

\section{Introdução}

\begin{frame}{\secname}
    \begin{itemize}
        \item<1-> Motivação
        \only<1-1>{
            \begin{itemize}
                \item Popularização do voto pela internet
                \item Praticidade e redução de custos
                \item Evitar segurança por obscuridade
            \end{itemize}

            \note{%
                Este trabalho tem como motivação popularizar o voto pela
                internet por uma alternativa, usando a computação como forma de
                aumentar a privacidade e reduzir os custos de eleições, e
                diferente dos sistemas eleitorais usados hoje em dia evitar
                obter segurança por obscuridade, usando algoritmos livres com
                bastante estudo na literatura.
            }
        }

        \item<2-> Justificativa
        \only<2-2>{
            \begin{itemize}
                \item Falta de conhecimento da população
                \item Digitização de sistemas em geral
            \end{itemize}

            \note{%
                Como justificativa, notou-se desconhecimento generalizado da
                população acerca de meios eletrônicos para votação, como no
                caso das urnas eletrônicas mesmo após 20 anos da instalação.
                Além disso, há uma tendência de digitização de sistemas em
                geral, como prestação de serviços governamentais.
            }
        }

        \item<3-> Objetivos
        \only<3-3>{
            \begin{itemize}
                \item Estudar uma solução de eleição remota verificável (ADDER)
                \item Analisar trabalhos correlatos na literatura
                \item Explorar implicações em segurança e privacidade do
                    processo
            \end{itemize}

            \note{%
                Os objetivos são estudar um protocolo de eleição remota
                verificável chamado ADDER, cujo desenvolvimento se deu em 2007,
                analisar em conjunto trabalhos correlatos da literatura e
                explorar as implicações em segurança e privacidade que cada
                decisão impõe.
            }
        }
    \end{itemize}
\end{frame}

\section{Evolução dos sistemas de eleição}

\begin{frame}{\secname}
    \begin{itemize}
        \item<1-> Voto por aclamação
        \only<1-1>{
            \begin{itemize}
                \item Nenhum mecanismo de privacidade
                \item Perfeita auditoria do resultado
            \end{itemize}

            \note{%
                O primeiro sistema que se conhece é o voto por aclamação, onde
                cada eleitor anunciava seu voto em voz alta. Esse sistema,
                notadamente, não tem nenhum mecanismo de privacidade, mas
                permite uma auditoria perfeita do resultado uma vez que
                qualquer eleitor pode anotar os votos e conferir o resultado.
                Perceberemos que a perfeita confiabilidade do resultado e a
                perfeita privacidade do voto são incompatíveis.
            }
        }

        \item<2-> Cédulas de papel
        \only<2-2>{
            \begin{itemize}
                \item Possibilidade de sigilo do voto
                \item Dificuldade para garantir confiabilidade
                \item Acessibilidade com cédulas impressas
            \end{itemize}

            \note{%
                Como forma de buscar algum sigilo, foram introduzidas as
                cédulas de papel. Era mais difícil auditar o resultado, e esse
                sistema foi usado por tanto tempo que foram desenvolvidas
                inúmeras fraudes. As cédulas com candidados impressos em vez de
                em branco auxiliaram na acessibilidade das eleições uma vez que
                não era mais necessário saber escrever, mas se tornou
                necessário saber ler.
            }
        }

        \item<3-> Máquinas mecânicas e eletrônicas
        \only<3-3>{
            \begin{itemize}
                \item Peças móveis causam falhas e travamentos
                \item Problemas no destaque de cartões perfurados
                \item Dependência no \textit{software} utilizado
            \end{itemize}

            \note{%
                Partimos então para as máquinas mecânicas e eletrônicas, que
                eram eficientes, reduziam custos, mas quebravam com facilidade
                por causa de partes móveis e outras falhas com os meios, como
                cartões perfurados que não destacavam corretamente. Também
                criavam um ponto de falha no software utilizado.
            }
        }

        \item<4-> Urnas eletrônicas
        \only<4-4>{
            \begin{itemize}
                \item \textbf{MUITA} dependência no \textit{software} utilizado
                \item Descuido no uso de criptografia
                \item Relatório de \textcite{aranha2012vulnerabilidades}
            \end{itemize}

            \note{%
                O que há de mais moderno no Brasil são as urnas eletrônicas,
                que dependem quase que totalmente do software utilizado, ainda
                mais que as anteriores. Além disso, os poucos relatórios que
                puderam ser publicados indicam graves falhas nos componentes
                criptográficos, beirando a inconstitucionalidade.
            }
        }

    \end{itemize}
\end{frame}

\section{Trabalhos correlatos}

\begin{frame}{\secname}
    \begin{itemize}
        \item<1-> ADDER
        \only<1-1>{
            \begin{itemize}
                \item Criptografia homomórfica (\textit{ElGamal})
                \item Chaves criptográficas distribuídas entre autoridades
                \item Não armazena informações secretas
            \end{itemize}

            \note{%
                O ADDER, principal influência deste trabalho, é um protocolo
                que utiliza criptografia homomórfica para armazenamento dos
                votos, que será detalhado adiante. Ele também define que as
                chaves criptográfias são distribuídas entre autoridades, sejam
                elas comissões eleitorais, partidos interessados ou até
                eleitores, de forma que o sistema sozinho não possa decifrar o
                resultado. É construído sobre um servidor que não armazena
                nenhuma informação secreta, somente votos criptografados e
                chaves públicas.
            }
        }

        \item<2-> Civitas
        \only<2-2>{
            \begin{itemize}
                \item Resistência à coerção e corrupção
                \item Compatível com diferentes tipos de votação
                \item Também distribui chaves
            \end{itemize}

            \note{%
                O Civitas é ligeiramente diferente por ser voltado a pesquisas
                em vez de eleições, e combate ativamente coerção e corrupção
                dos eleitores e autoridades. Ele possui outros tipos de votação
                além de voto majoritário. As chaves também são distribuídas
                entre participantes.
            }
        }

        \item<3-> Helios
        \only<3-3>{
            \begin{itemize}
                \item Integração com serviços de autenticação (\textit{CAS},
                    \textit{OAuth}, \textit{Shibboleth})
                \item Possibilidade de auditar cada passo
                \item Maior foco em usabilidade e intuitividade
            \end{itemize}

            \note{%
                O Helios é um sistema mais prático, que possui integração com
                serviços de login como CAS (usado na UFSC), OAuth (login por
                Google e Facebook), entre outros. Ele permite auditoria de
                todas as etapas, e o eleitor pode conferir se seu voto foi
                contado ao final. Ele é focado em usabilidade e intuitividade
                em vez de perfeita segurança.
            }
        }
    \end{itemize}
\end{frame}

\section{Implementação}

\begin{frame}{\secname}
    \begin{itemize}
        \item<1-> Autenticação terceirizada
        \only<1-1>{
            \begin{itemize}
                \item Serviços de OAuth e OpenID
                \item Separação de responsabilidades
            \end{itemize}

            \note{%
                A implementação proposta se inicia pelo serviço terceirizado de
                autenticação, usando protocolos como OAuth e OpenID para
                efetuar autenticação dos usuários de forma independente do
                servidor de eleição. Notadamente, serviços como Google e
                Facebook são suficientes para eleições médias, enquanto
                eleições governamentais podem usar serviços de certificados
                digitais para isso.
            }
        }

        \item<2-> Criptosistema de Paillier
        \only<2-2>{
            \begin{itemize}
                \item Algoritmo probabilístico
                    \begin{equation*}
                    \begin{aligned}
                        c &= g^m \cdot r^n \bmod{n^2} \\
                        m &= \frac{\left( c^\lambda \bmod{n^2} \right) - 1}{n} \cdot \mu \bmod{n}
                    \end{aligned}
                    \end{equation*}

                \item Criptografia homomórfica aditiva
                    \begin{equation*}
                    \begin{aligned}
                        D \left( E \left( m_1 \right) \cdot E \left( m_2 \right) \bmod{n^2} \right) &= m_1 + m_2 \bmod{n} \\
                        D \left( E \left( m_1 \right)^k \bmod{n^2} \right) &= k m_1 \bmod{n}
                    \end{aligned}
                    \end{equation*}
            \end{itemize}

            \note{%
                Os votos são criptografados utilizando o criptosistema de
                Paillier, que é probabilístico, ou seja, sucessivas cifragens
                da mesma combinação de votos geram cifras diferentes, evitando
                a possibilidade de se revelar um voto sem a chave privada. O
                criptosistema é homomórfico para adição, ou seja, é possível
                obter a soma de duas mensagens originais operando somente nas
                mensagens cifradas. Essa propriedade é usada para que o sistema
                possa obter um somatório dos votos sem decifrá-los um a um ao
                final da eleição.
            }
        }

        \item<3-> Compartilhamento de segredos de Shamir
        \only<3-3>{
            \begin{itemize}
                \item Divisão da chave privada entre $n$ autoridades
                \item Pelo menos $k$ autoridades precisam colaborar para recuperar a chave
            \end{itemize}

            \framebox{\small $k$ pontos definem polinômio de grau $k-1$}

            \note{%
                O compartilhamento de segredos de Shamir é usado para dividir
                uma chave privada entre n autoridades de forma que um
                subconjunto k menor ou igual a n das autoridades possa
                reestabeler a chave privada original. Dessa forma, um agressor
                precisa corromper ao menos k autoridades para obter vantagem no
                processo, e esse k pode ser configurado para que isso seja
                improvável de acontecer.
            }
        }

        \item<4-> HTTPS, CSP, HSTS, HPKP, SRI, \ldots
        \only<4-4>{
            \note{%
                Além disso, medidas de segurança e privacidade do protocolo
                HTTP são naturalmente adotadas, como HTTPS para proteger o
                canal de votação, HSTS e HPKP para evitar o vazamento de
                certificados digitais, CSP e SRI para evitar injeção de
                conteúdo malicioso na interface, entre outros.
            }
        }
    \end{itemize}
\end{frame}

\section{Protocolo}

\begin{frame}{\secname}
    \begin{enumerate}
        \item Enviar os parâmetros da nova eleição
        \item Envio das chaves privadas das autoridades
        \item Geração e distribuição do par de chaves
        \item Coleta de votos
        \item Totalização dos votos cifrados
        \item Decifragens parciais pelas autoridades
        \item Combinação e publicação dos resultados
    \end{enumerate}

    \framebox{\small Simplificação do protocolo ADDER}

    \note{%
        O protocolo usado é uma simplificação do ADDER, uma vez que o ADDER
        utiliza o criptosistema de ElGamal que tem mais etapas na geração de
        chaves. <Ler passos>
    }
\end{frame}

\section{Trabalhos futuros}

\begin{frame}{\secname}
    \begin{itemize}
        \item<1-> Infraestrutura de chaves públicas
        \only<1-1>{
            \note{%
                Uma possível melhoria nesta implementação é o uso de uma
                infraestrutura de chaves públicas para gerenciar as chaves das
                autoridades e reforçar a autenticação dos usuários.
            }
        }

        \item<2-> Remoção da responsabilidade do administrador do sistema
        \only<2-2>{
            \begin{itemize}
                \item Mudança de paradigma
            \end{itemize}

            \note{%
                A figura de administrador do sistema poderia ser substituída
                por um consenso entre as autoridades, de forma que o
                administrador não seja mais diretamente necessário nas
                eleições. Isso geraria uma mudança de paradigma, uma vez que
                eleições em geral dependem de uma comissão eleitoral (como o
                TSE) que administram o procedimento.
            }
        }

        \item<3-> \textit{Blockchain}
        \only<3-3>{
            \begin{itemize}
                \item Consenso descentralizado dos valores
            \end{itemize}

            \note{%
                O uso da tecnologia de blockchain pode descentralizar o
                consenso dos votos, de forma que em vez de corromper
                autoridades, um agressor precise corromper ao menos metade do
                total de eleitores, tornando virtualmente impossível manipular
                os resultados. É necessário cuidar para evitar a possibilidade
                de traçar o voto de um eleitor específico.
            }
        }

        \item<4-> Verificação formal
        \only<4-4>{
            \note{%
                Embora não tenha sido proposto nenhum novo algoritmo na
                elaboração desse trabalho, uma verificação formal do
                funcionamento conjunto das partes garantiria confiabilidade nos
                resultados de uma eleição nos moldes dessa implementação.
            }
        }
    \end{itemize}
\end{frame}

\maketitle

\end{document}
