\documentclass[14pt]{beamer}
\usetheme[
    numbering=none,
    progressbar=frametitle,
    sectionpage=none,
]{metropolis}

\usepackage{csquotes,polyglossia}
\setdefaultlanguage{brazil}

\usepackage[style=abnt, language=brazil]{biblatex}
\addbibresource{biblio.bib}

\title{Implementação de um sistema de eleição remoto secreto e verificável}
\author{Ranieri Schroeder Althoff \\ Orientadora: Taciane Martimiano}
\date{\today}
\institute{%
    Universidade Federal de Santa Catarina -- UFSC
    \par
    Departamento de Informática e Estatística -- INE
    \par
    Programa de Graduação em Ciências da Computação}

\begin{document}

\maketitle

\section{Introdução}

\begin{frame}{\secname}
    \begin{itemize}
        \item<1-> Motivação
        \only<1-1>{
            \begin{itemize}
                \item Popularização do voto pela internet
                \item Praticidade e redução de custos
                \item Evitar segurança por obscuridade
            \end{itemize}
        }

        \item<2-> Justificativa
        \only<2-2>{
            \begin{itemize}
                \item Falta de conhecimento da população
                \item Digitização de sistemas em geral
            \end{itemize}
        }

        \item<3-> Objetivos
        \only<3-3>{
            \begin{itemize}
                \item Estudar uma solução de eleição remota verificável (ADDER)
                \item Analisar trabalhos correlatos na literatura
                \item Explorar implicações em segurança e privacidade do
                    processo
            \end{itemize}
        }
    \end{itemize}
\end{frame}

\section{Evolução dos sistemas de eleição}

\begin{frame}{\secname}
    \begin{itemize}
        \item<1-> Voto por aclamação
        \only<1-1>{
            \begin{itemize}
                \item Nenhum mecanismo de privacidade
                \item Perfeita auditoria do resultado
            \end{itemize}
        }

        \item<2-> Cédulas de papel
        \only<2-2>{
            \begin{itemize}
                \item Possibilidade de sigilo do voto
                \item Dificuldade para garantir confiabilidade
                \item Acessibilidade com cédulas impressas
            \end{itemize}
        }

        \item<3-> Máquinas mecânicas e eletrônicas
        \only<3-3>{
            \begin{itemize}
                \item Peças móveis causam falhas e travamentos
                \item Problemas no destaque de cartões perfurados
                \item Dependência no \textit{software} utilizado
            \end{itemize}
        }

        \item<4-> Urnas eletrônicas
        \only<4-4>{
            \begin{itemize}
                \item \textbf{MUITA} dependência no \textit{software} utilizado
                \item Descuido no uso de criptografia
                \item Relatório de \textcite{aranha2012vulnerabilidades}
            \end{itemize}
        }

    \end{itemize}
\end{frame}

\section{Trabalhos correlatos}

\begin{frame}{\secname}
    \begin{itemize}
        \item<1-> ADDER
        \only<1-1>{
            \begin{itemize}
                \item Criptografia homomórfica (\textit{ElGamal})
                \item Chaves criptográficas distribuídas entre autoridades
                \item Não armazena informações secretas
            \end{itemize}
        }

        \item<2-> Civitas
        \only<2-2>{
            \begin{itemize}
                \item Resistência à coerção e corrupção
                \item Compatível com diferentes tipos de votação
                \item Também distribui chaves
            \end{itemize}
        }

        \item<3-> Helios
        \only<3-3>{
            \begin{itemize}
                \item Integração com serviços de autenticação (\textit{CAS},
                    \textit{OAuth}, \textit{Shibboleth})
                \item Possibilidade de auditar cada passo
                \item Maior foco em usabilidade e intuitividade
            \end{itemize}
        }
    \end{itemize}
\end{frame}

\section{Implementação}

\begin{frame}{\secname}
    \begin{itemize}
        \item<1-> Autenticação terceirizada
        \only<1-1>{
            \begin{itemize}
                \item Serviços de OAuth e OpenID
                \item Separação de responsabilidades
            \end{itemize}
        }

        \item<2-> Criptosistema de Paillier
        \only<2-2>{
            \begin{itemize}
                \item Algoritmo probabilístico
                    \begin{equation*}
                    \begin{aligned}
                        c &= g^m \cdot r^n \bmod{n^2} \\
                        m &= \frac{\left( c^\lambda \bmod{n^2} \right) - 1}{n} \cdot \mu \bmod{n}
                    \end{aligned}
                    \end{equation*}

                \item Criptografia homomórfica aditiva
                    \begin{equation*}
                    \begin{aligned}
                        D \left( E \left( m_1 \right) \cdot E \left( m_2 \right) \bmod{n^2} \right) &= m_1 + m_2 \bmod{n} \\
                        D \left( E \left( m_1 \right)^k \bmod{n^2} \right) &= k m_1 \bmod{n}
                    \end{aligned}
                    \end{equation*}
            \end{itemize}
        }

        \item<3-> Compartilhamento de segredos de Shamir
        \only<3-3>{
            \begin{itemize}
                \item Divisão da chave privada entre $n$ autoridades
                \item Pelo menos $k$ autoridades precisam colaborar para recuperar a chave
            \end{itemize}

            \framebox{\small $k$ pontos definem polinômio de grau $k-1$}
        }

        \item<4-> HTTPS, CSP, HSTS, HPKP, SRI, \ldots
    \end{itemize}
\end{frame}

\section{Protocolo}

\begin{frame}{\secname}
    \begin{enumerate}
        \item Enviar os parâmetros da nova eleição
        \item Envio das chaves privadas às autoridades
        \item Geração e distribuição do par de chaves
        \item Coleta de votos
        \item Totalização dos votos cifrados
        \item Decifragens parciais pelas autoridades
        \item Combinação e publicação dos resultados
    \end{enumerate}

    \framebox{\small Simplificação do protocolo ADDER}
\end{frame}

\section{Trabalhos futuros}

\begin{frame}{\secname}
    \begin{itemize}
        \item<1-> Infraestrutura de chaves públicas

        \item<2-> Remoção da responsabilidade do administrador do sistema
        \only<2-2>{
            \begin{itemize}
                \item Mudança de paradigma
            \end{itemize}
        }

        \item<3-> \textit{Blockchain}
        \only<3-3>{
            \begin{itemize}
                \item Consenso descentralizado dos valores
            \end{itemize}
        }

        \item<4-> Verificação formal
    \end{itemize}
\end{frame}

\end{document}
